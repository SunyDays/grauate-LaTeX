\documentclass[aspectratio=169]{beamer}

\usepackage[utf8]{inputenc}
\usepackage[russianb]{babel}
\usepackage{amssymb}
\usepackage{amsmath}
\usepackage{moreverb}

% flowcharts
\usepackage{tikz}
\usetikzlibrary{shapes,arrows,
	decorations.pathreplacing,decorations.pathmorphing}
	
% for tables
\usepackage{multirow}
\usepackage{hhline}

% include pictures
\usepackage{graphicx}
\DeclareGraphicsExtensions{.png}
\graphicspath{ {pics/} }	

\title{Разработка программной среды аналитического моделирования практико-ориентированных информационных систем}
\author[Лакеев Р.Д.]{Лакеев Роман}
\institute[МГТУ <<СТАНКИН>>]{ФГБОУ ВПО МГТУ <<СТАНКИН>>}
\date{Москва, 2015}

\usetheme{default}
\usecolortheme{default}
\usefonttheme{professionalfonts}

\begin{document}

\maketitle

\begin{frame}{Введение}
\begin{block}{Цель}
Анализ критериев времени и надёжности доставки информации в информационно-вычислительных сетях с множественным методом доступа без коллизий, построенных на основе технологий семейства Ethernet.	
\end{block}

\begin{block}{Задачи}
\begin{enumerate}
	\item Изучение методики разработки моделей сетей.
	\item Разработка аналитических математических моделей ИВС.
	\item Разработка программы для вычисления стационарных и интегральных вероятностных характеристик заданной ИВС.
	\item Проведение модельного эксперимента.
\end{enumerate}
\end{block}
\end{frame}

\begin{frame}{Введение}
\begin{block}{Методы}
Модельный эксперимент и математические модели фрагментов сетей
основываются на математическом аппарате и методах теории систем и сетей массового обслуживания.
\end{block}

\begin{block}{Значимость}
Разработанная программа должна автоматизировать рутинную работу по вычислению стационарных и интегральных вероятностных характеристик, плотностей распределения сообщений в маршрутах сети и среднего количества маршрутов между любыми двумя узлами сети.
\end{block}

%\begin{block}{}
%\end{block}
\end{frame}

\begin{frame}{Теоретические основы}
\begin{block}{Сети массового обслуживания}
СеМО представляет собой совокупность конечного числа \( M \) обслуживающих центров, в которой циркулируют сообщения,
переходящие в соответствии с маршрутной матрицей из одного центра сети в другой.
Центром обслуживания является система массового обслуживания, состоящая из \( A \; (1 \leqslant A \leqslant \infty) \) одинаковых приборов
и буфера объёмом \( C \; (0 \leqslant C \leqslant \infty) \). Если в момент поступления сообщения все обслуживающие приборы центра заняты, то сообщение занимает очередь в буфере и ожидает обслуживания.
\end{block}
\end{frame}

\begin{frame}{Теоретические основы}
\begin{block}{Однородные экспоненциальные сети}
В данной работе рассматриваются открытые сети Джексона с бесконечным буфером, обрабатывающие \( F \) входящих потоков.

Сеть Джексона это СеМО, в которой время обслуживания заявок распределено по экспоненциальному закону, а распределение входящего потока имеет распределение Пуассона. Такая модель даёт верхнюю границу оценки (худший вариант)
и стационарные вероятности состояний сети имеют мультипликативную форму.
\end{block}	
\end{frame}

\begin{frame}{Теоретические основы}
\begin{block}{Пуассоновский поток}
\begin{enumerate}
	\item Стационарность --- вероятность появления \( k \) событий на любом промежутке времени зависит только от числа \( k \) и от длительности
	\( t \) промежутка.
	
	\item Ординарность --- вероятность наступления за элементарный промежуток времени более одного события мала по сравнению с вероятностью
	наступления за этот промежуток не более одного события и ей можно пренебречь.
	
	\item Независимость --- вероятность появления \( k \) на любом промежутке времени не зависит от того, появлялись или не появлялись
	события в моменты времени, предшествующие началу рассматриваемого промежутка.
\end{enumerate}
\end{block}
\end{frame}

\begin{frame}{Теоретические основы}
\begin{block}{Маршрутная матрица}
Маршрутная матрица задаёт топологию сети и вероятности переходов сообщения между узлами.
Для открытой сети в качестве внешнего источника вводится новый узел с индексом \( 0 \).
\end{block}
\end{frame}

\end{document}