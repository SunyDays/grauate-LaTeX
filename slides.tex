\documentclass[aspectratio=169]{beamer}

\usepackage[utf8]{inputenc}
\usepackage[russianb]{babel}
\usepackage{amssymb}
\usepackage{amsmath}
\usepackage{moreverb}

% flowcharts
\usepackage{tikz}
\usetikzlibrary{shapes,arrows,
	decorations.pathreplacing,decorations.pathmorphing}
	
% for tables
\usepackage{multirow}
\usepackage{hhline}

% include pictures
\usepackage{graphicx}
\DeclareGraphicsExtensions{.png}
\graphicspath{ {pics/} }	

\title{Разработка программной среды аналитического моделирования практико-ориентированных информационных систем}
\author[Лакеев Р.Д.]{Лакеев Роман}
\institute[МГТУ <<СТАНКИН>>]{ФГБОУ ВПО МГТУ <<СТАНКИН>>}
\date{Москва, 2015}

\usetheme{default}
\usecolortheme{default}
\usefonttheme{professionalfonts}

\begin{document}

\maketitle

\begin{frame}
\frametitle{Введение}

\begin{block}{Цель}
Анализ критериев времени и надёжности доставки информации в информационно-вычислительных сетях с множественным методом доступа без коллизий, построенных на основе технологий семейства Ethernet.	
\end{block}

\begin{block}{Задачи}
\begin{enumerate}
	\item Изучение методики разработки моделей сетей.
	\item Разработка аналитических математических моделей ИВС.
	\item Разработка программы для вычисления стационарных и интегральных вероятностных характеристик заданной ИВС.
	\item Проведение модельного эксперимента.
\end{enumerate}
\end{block}
\end{frame}

\begin{frame}
\frametitle{Введение}

\begin{block}{Методы}
Модельный эксперимент и математические модели фрагментов сетей
основываются на математическом аппарате и методах теории систем и сетей массового обслуживания.
\end{block}

\begin{block}{Значимость}
Разработанная программа должна автоматизировать рутинную работу по вычислению стационарных и интегральных вероятностных характеристик, плотностей распределения сообщений в маршрутах сети и среднего количества маршрутов между любыми двумя узлами сети.
\end{block}

\end{frame}

\begin{frame}
\frametitle{Теоретические основы}

\begin{block}{Сети массового обслуживания}
СеМО представляет собой совокупность конечного числа \( M \) обслуживающих центров, в которой циркулируют сообщения,
переходящие в соответствии с маршрутной матрицей из одного центра сети в другой.
Центром обслуживания является система массового обслуживания, состоящая из \( A \; (1 \leqslant A \leqslant \infty) \) одинаковых приборов
и буфера объёмом \( C \; (0 \leqslant C \leqslant \infty) \). Если в момент поступления сообщения все обслуживающие приборы центра заняты, то сообщение занимает очередь в буфере и ожидает обслуживания.
\end{block}
\end{frame}

\begin{frame}
\frametitle{Теоретические основы}

\begin{block}{Однородные экспоненциальные сети}
В данной работе рассматриваются открытые сети Джексона с бесконечным буфером, обрабатывающие \( F \) входящих потоков.

Сеть Джексона это СеМО, в которой время обслуживания заявок распределено по экспоненциальному закону, а распределение входящего потока имеет распределение Пуассона. Такая модель даёт верхнюю границу оценки (худший вариант)
и стационарные вероятности состояний сети имеют мультипликативную форму.
\end{block}	
\end{frame}

\begin{frame}
\frametitle{Теоретические основы}
\framesubtitle{Однородные экспоненциальные сети}

\begin{block}{Пуассоновский поток}
\begin{enumerate}
	\item Стационарность --- вероятность появления \( k \) событий на любом промежутке времени зависит только от числа \( k \) и от длительности
	\( t \) промежутка.
	
	\item Ординарность --- вероятность наступления за элементарный промежуток времени более одного события мала по сравнению с вероятностью
	наступления за этот промежуток не более одного события и ей можно пренебречь.
	
	\item Независимость --- вероятность появления \( k \) на любом промежутке времени не зависит от того, появлялись или не появлялись
	события в моменты времени, предшествующие началу рассматриваемого промежутка.
\end{enumerate}
\end{block}
\end{frame}

\begin{frame}
\frametitle{Теоретические основы}
\framesubtitle{Однородные экспоненциальные сети}

\begin{block}{Маршрутная матрица}
Маршрутная матрица задаёт топологию сети и вероятности переходов сообщения между узлами.
Для открытой сети в качестве внешнего источника вводится новый узел с индексом \( 0 \).
\[ P^{m} = \left(\begin{matrix}
                0 			& P_{0,1}^{m}	& \cdots 	& P_{0,j}^{m} \\
                P_{1,0}^{m} 	& 0				& \cdots		& P_{1,j} \\
                \vdots 		& \vdots			& \ddots		& \vdots  \\
                P_{i,0}^{m} 	& P_{i,1}^{m} 	& \cdots		& 0 \\
            \end{matrix}\right), \; i,j = \overline{1,n}, \; n - \text{число узлов в сети} \]
\end{block}
\end{frame}

\begin{frame}
\frametitle{Теоретические основы}

\begin{block}{Методика разработки математической модели ИВС}
В работе используется приближённая декомпозиционная модель сети массового обслуживания.

Методика включает в себя разработку отдельных математических моделей всех составляющих ИВС на всех уровнях декомпозиции и состоит из следующих подэтапов:
\begin{enumerate}
	\item  Составление уравнений баланса интенсивностей потоков.
	\item  Вычислеие коэффициентов передачи из уравнений баланса.
	\item  Вычисление стационарных вероятностно-временных характеристик \\ (ВВХ) для каждого отдельного элемента СеМО.
	\item  Вычисиление интегральных ВВХ при взаимодействии двух любых абонентов сети.
\end{enumerate}
\end{block}
\end{frame}

\begin{frame}
\frametitle{Теоретические основы}
\framesubtitle{Методика разработки математической модели ИВС}
	
\begin{block}{Исходные параметры}
Исходными параметрами модели являются интенсивности обслуживающих узлов сети \( \mu_{i}^{m} \), интенсивности поступления сообщений из внешнего источника
\( \lambda_{i}^{m} \) и маршрутная матрица \( P^{m} \) для каждого входного потока \( m = \overline{1, F} \).
\end{block}
\end{frame}

\begin{frame}
\frametitle{Теоретические основы}
\framesubtitle{Методика разработки математической модели ИВС}

\begin{block}{Уравнения баланса}
Уравнения баланса позволяют найти общие интенсивности потоков \( \lambda_{i}^{'m} \) сообщений в стационарном режиме открытой СеМО.

\[ \lambda_{i}^{'m} = e_{i}^{m} \lambda_{0}^{m} \]
\( e_{i}^{m} \) - коэффициенты передачи, получаемые при решении уравнений баланса,
\( \lambda_{0}^{m} = \sum\limits_{i = 1}^{n} \lambda_{i}^{m}\) - суммарная интенсивность всех внешних потоков типа \( m \).

Уравнения баланса имеют следующий вид:

\[ \left\{
		\begin{aligned}
			& e_{1}^{m} = P_{01}^{m} + e_{1}^{m} P_{11}^{m} + \cdots + e_{n}^{m} P_{n1}^{m} \\
			& \vdots \\
			& e_{n}^{m} = P_{0n}^{m} + e_{1}^{m} P_{1n}^{m} + \cdots + e_{n}^{m} P_{nn}^{m}
		\end{aligned}
	\right. \]
\end{block}
\end{frame}

\begin{frame}
\frametitle{Теоретические основы}
\framesubtitle{Методика разработки математической модели ИВС}

\begin{block}{Коэффициенты загрузки}
Коэффициент загрузки для узла \( M_{i} \) вычисляется по формуле
\[ \rho_{i} = \sum\limits_{m = 0}^{F} \rho_{i}^{m}, \: \rho_{i}^{m} = \frac{\lambda_{i}^{'m}}{\mu_{i}^{m}}, \:
i = \overline{1, n} \]
\( \rho_{i}^{m} \)- коэффициент использования узла. Для существования стационарного распределения числа сообщений в системе необходимо выполнение условия
\[ 0 \leqslant \rho_{i}, \rho_{i}^{m} \leqslant 1, \: i = \overline{1, n} \]
\end{block}
\end{frame}

\begin{frame}
\frametitle{Теоретические основы}
\framesubtitle{Методика разработки математической модели ИВС}

\begin{block}{Стационарные вероятностно-временные характеристики}
Для каждого узла  \( M_{i} \) сети определяются следующие вероятностно-временные характеристики:

\begin{columns}[t]
\column{.5\textwidth}
\begin{enumerate}
\item Средняя длительность ожидания обслуживания.
	\[
		W_{i} = \frac{ \sum\limits_{m = 0}^{F} \frac{ \rho_{i}^{m} }{ \mu_{i}^{m}} }{ 1 - \rho_{i} }
	\]
	
	\item Средняя длительность пребывания сообщения в узле для потока \( m \).
	\[ U_{i}^{m} = W_{i} + \frac{1}{\mu_{i}^{m}} \]
\end{enumerate}

\column{.5\textwidth}
\begin{enumerate}
\setcounter{enumi}{2}

	\item Средняя длина очереди сообщений в узле для потока \( m \).
	\[ L_{i}^{m} = \lambda_{i}^{'m} W_{i} \]
	
	\item Среднее число сообщений в узле для потока \( m \)
	\[ N_{i}^{m} = \lambda_{i}^{'m} U_{i}^{m} \]
\end{enumerate}
\end{columns}
\end{block}
\end{frame}

\begin{frame}
\frametitle{Теоретические основы}
\framesubtitle{Методика разработки математической модели ИВС}

\begin{block}{Интегральные вероятностно-временные характеристики}
Для определения интегральных ВВХ используются стационарные ВВХ, полученные для каждого узла сети, и анализ маршрутов движения сообщений между двумя абнентами
\( A_{i} \) и \( A_{j}, \: i \neq j \).

Любой маршрут между двумя любыми абонентами принадлежит к одному из следующих типов:
\begin{enumerate}
\item Последовательная обработка сообщений на конечном числе элементов сети.

\begin{figure}[h!]
		\begin{center}
			\begin{picture}(200, 20)(-10, -10)
				\put(0, 0){\circle{20}}
				\put(-5, -3){\begin{scriptsize} 1 \end{scriptsize}}
	
				\put(10, 0){\vector(1, 0){20}}
				
				\put(40, 0){\circle{20}}
				\put(35, -3){\begin{scriptsize} 2 \end{scriptsize}}
		
				\put(50, 0){\vector(1, 0){20}}
				
				\put(78, -1){\( \dots \)}

				\put(102, 0){\vector(1, 0){20}}
				
				\put(132, 0){\circle{20}}
				\put(119.5, -3){\begin{scriptsize} \( n - 1 \)\end{scriptsize}}
				
				\put(142, 0){\vector(1, 0){20}}
				
				\put(172, 0){\circle{20}}
				\put(166.5, -2.5){\begin{scriptsize} \( n \)\end{scriptsize}}
			\end{picture}
		\end{center}
	\end{figure}
	
	\[ W = \sum\limits_{i = 1}^{n} W_{i}, \;
	U^{m} = \sum\limits_{i = 1}^{n} U_{i}^{m}, \;
	L^{m} = \sum\limits_{i = 1}^{n} L_{i}^{m}, \;
	N^{m} = \sum\limits_{i= 1}^{n} N_{i}^{m} \]	
\end{enumerate}
\end{block}
\end{frame}

\begin{frame}

\begin{enumerate}
\setcounter{enumi}{1}
\item Параллельные варианты обработки с определёнными значениями вероятности выбора рассматриваемого варианта.

\begin{columns}[t]
\column{.5\textwidth}
\begin{figure}[h!]
		\begin{center}
			\begin{picture}(220, 120)
				\put(10, 50){\circle{20}}
				\put(5.5, 47){\begin{scriptsize} \( i \) \end{scriptsize}}
				
				\put(70, 100){\begin{scriptsize} \( \beta_{1} \) \end{scriptsize}}
				\put(20, 50){\vector(4, 3){80}}
				
				\put(110, 110){\circle{20}}
				\put(105, 107){\begin{scriptsize} 1 \end{scriptsize}}
				
				\put(120, 110){\vector(4, -3){80}}
				
				\put(70, 70){\begin{scriptsize} \( \beta_{2} \) \end{scriptsize}}
				\put(20, 50){\vector(4, 1){80}}
				
				\put(110, 70){\circle{20}}
				\put(105, 67){\begin{scriptsize} 2 \end{scriptsize}}
				
				\put(120, 70){\vector(4, -1){80}}
				
				\put(20, 50){\vector(4, -1){80}}				
				\put(108, 35){\( \vdots \)}
				
				\put(70, 25){\begin{scriptsize} \( \beta_{l} \) \end{scriptsize}}
				\put(20, 50){\vector(2, -1){80}}
				
				\put(110, 10){\circle{20}}
				\put(105.5, 7){\begin{scriptsize} \( l \) \end{scriptsize}}
				
				\put(120, 10){\vector(2, 1){80}}
				
				\put(210, 50){\circle{20}}
				\put(205.5, 47){\begin{scriptsize} \( j \) \end{scriptsize}}
			\end{picture}
		\end{center}
	\end{figure}
	
\column{.5\textwidth}
\[ W = W_{i} + \sum\limits_{k = 1}^{l} \beta_{k} W_{k} + W_{j} \]
\[ U^{m} = U_{i}^{m} + \sum\limits_{k = 1}^{l} \beta_{k} U_{k}^{m} + U_{j}^{m} \]
\[ L^{m} = L_{i}^{m} + \sum\limits_{k = 1}^{l} \beta_{k} L_{k}^{m} + L_{j}^{m} \]
\[ N^{m} = N_{i}^{m} + \sum\limits_{k = 1}^{l} \beta_{k} N_{k}^{m} + N_{j}^{m} \]
\end{columns}
\end{enumerate}
\end{frame}

\begin{frame}

\begin{enumerate}
\setcounter{enumi}{2}
\item Комбинация первых двух типов.

\begin{figure}[h!]
		\begin{center}
			\begin{picture}(382, 120)
				\put(10, 50){\circle{20}}
				\put(5.5, 47){\begin{scriptsize} \( i \) \end{scriptsize}}
				
				\put(70, 100){\begin{scriptsize} \( \beta_{1} \) \end{scriptsize}}
				\put(20, 50){\vector(4, 3){80}}
				
				%--------------------------------------------------------------------------
					\put(110, 110){\circle{20}}
					\put(105, 107){\begin{scriptsize} 1 \end{scriptsize}}
	
					\put(120, 110){\vector(1, 0){15}}
%				
%					\put(145, 110){\circle{20}}
%					\put(140, 107){\begin{scriptsize} 2 \end{scriptsize}}
		
%					\put(155, 110){\vector(1, 0){15}}

					\put(138, 109){\( \dots \)}

					\put(154, 110){\vector(1, 0){15}}
				
%					\put(237, 110){\circle{20}}
%					\put(224.5, 107){\begin{scriptsize} \( n - 1 \)\end{scriptsize}}
%					
%					\put(247, 110){\vector(1, 0){15}}
				
					\put(179, 110){\circle{20}}
					\put(174.5, 107.5){\begin{scriptsize} \( n \)\end{scriptsize}}	
				%--------------------------------------------------------------------------
				
				\put(189, 110){\vector(4, -3){80}}
				
				\put(70, 70){\begin{scriptsize} \( \beta_{2} \) \end{scriptsize}}
				\put(20, 50){\vector(4, 1){80}}
				
				%--------------------------------------------------------------------------					
					\put(110, 70){\circle{20}}
					\put(105, 67){\begin{scriptsize} 1 \end{scriptsize}}
	
					\put(120, 70){\vector(1, 0){15}}
%				
%					\put(145, 70){\circle{20}}
%					\put(140, 67){\begin{scriptsize} 2 \end{scriptsize}}
		
%					\put(155, 70){\vector(1, 0){15}}

					\put(138, 69){\( \dots \)}

					\put(154, 70){\vector(1, 0){15}}
				
%					\put(237, 70){\circle{20}}
%					\put(224.5, 67){\begin{scriptsize} \( n - 1 \)\end{scriptsize}}
%					
%					\put(247, 70){\vector(1, 0){15}}
				
					\put(179, 70){\circle{20}}
					\put(174.5, 67.5){\begin{scriptsize} \( n \)\end{scriptsize}}					
				%--------------------------------------------------------------------------
				
				\put(189, 70){\vector(4, -1){80}}
				
				\put(20, 50){\vector(4, -1){80}}				
				\put(108, 35){\( \vdots \)}
				
				\put(70, 25){\begin{scriptsize} \( \beta_{l} \) \end{scriptsize}}
				\put(20, 50){\vector(2, -1){80}}
				
				%--------------------------------------------------------------------------					
					\put(110, 10){\circle{20}}
					\put(105, 7){\begin{scriptsize} 1 \end{scriptsize}}
	
					\put(120, 10){\vector(1, 0){15}}
%				
%					\put(145, 10){\circle{20}}
%					\put(140, 7){\begin{scriptsize} 2 \end{scriptsize}}
		
%					\put(155, 10){\vector(1, 0){15}}

					\put(138, 9){\( \dots \)}

					\put(154, 10){\vector(1, 0){15}}
				
%					\put(237, 10){\circle{20}}
%					\put(224.5, 7){\begin{scriptsize} \( n - 1 \)\end{scriptsize}}
%					
%					\put(247, 10){\vector(1, 0){15}}
				
					\put(179, 10){\circle{20}}
					\put(174.5, 7.5){\begin{scriptsize} \( n \)\end{scriptsize}}					
				%--------------------------------------------------------------------------
				
				\put(189, 10){\vector(2, 1){80}}
				
				\put(279, 50){\circle{20}}
				\put(274.5, 47){\begin{scriptsize} \( j \) \end{scriptsize}}
			\end{picture}
		\end{center}
	\end{figure}	

\[ W = W_{i} + \sum\limits_{k = 1}^{l} \left( \beta_{k} \sum\limits_{x = 1}^{n} W_{x} \right) + W_{j}, \; U^{m} = U_{i}^{m} + \sum\limits_{k = 1}^{l} \left( \beta_{k} \sum\limits_{x = 1}^{n} U_{x}^{m} \right) + U_{j}^{m}, \]
\[ L^{m} = L_{i}^{m} + \sum\limits_{k = 1}^{l} \left( \beta_{k} \sum\limits_{x = 1}^{n} L_{x}^{m} \right) + L_{j}^{m}, \; N^{m} = N_{i}^{m} + \sum\limits_{k = 1}^{l} \left( \beta_{k} \sum\limits_{x = 1}^{n} N_{x}^{m} \right) + N_{j}^{m} \]
\end{enumerate}
\end{frame}

\begin{frame}
\frametitle{Теоретические основы}
\framesubtitle{Методика разработки математической модели ИВС}

\begin{block}{Вероятность выбора маршрута}
Для вычисления вероятности выбора альтернативных маршрутов нужно учитывать вероятности перехода между узлами, заданные маршрутной матрицей \( P^{m} \), и нормирующее условие
\( \sum\limits_{k = 1}^{l} \beta_{k} = 1 \).

Вероятность выбора маршрута определяется отношением произведения вероятностей перехода требований из узла \( M_{ R_{j}^{i} } \) в узел \( M_{ R_{j+1}^{i} } \)
\( i \) - го маршрута к сумме произведений вероятностей переходов требований всех альтернативных маршрутов (нормировочной величине).
\[ \beta_{i} = \frac{ \prod\limits_{j} P_{ R_{j}^{i}, R_{j + 1}^{i} } }{ \sum\limits_{i} \left( \prod\limits_{j} P_{ R_{j}^{i}, R_{j+1}^{i} } \right) } \]
\end{block}
\end{frame}

\begin{frame}
\frametitle{Теоретические основы}
\framesubtitle{Методика разработки математической модели ИВС}

\begin{block}{Плотность распределения количества сообщений в маршруте}
Плотность распределения количества сообщений для произвольного \\ маршрута определяется следующим способом:
\[ g_{i} (t) = \sum\limits_{i = 1}^{n} H_{i} (\mu_{i} - \lambda_{i}^{'}) e^{ -(\mu_{i} - \lambda_{i}^{'}) t } \]
\[ H_{i} = \prod\limits_{ \substack{ j = 1 \\ j \neq i } }^{n} \frac{ \mu_{j} - \lambda_{j}^{'} }{ \mu_{j} - \lambda_{j}^{'} - \mu_{i} - \lambda_{i}^{'} } \]
где \( n \) - количество узлов в маршруте.
\end{block}
\end{frame}

\begin{frame}
\frametitle{Теоретические основы}
\framesubtitle{Методика разработки математической модели ИВС}

\begin{block}{Определение интенсивностей обслуживающих приборов, работающих на основе технологий семейства Ethernet}
Интенсивности обслуживания \( \mu \) определяются для кадров формата Ethernet Version 2.

\begin{table}[h!]
\begin{tabular}{|c|c|c|c|c|c|}
\hline Preamble & DA & SA & EthernetType & Data & Checksum \\
\hline 8 & 6 & 6 & 2 & 46 - 1500 & 4 \\
\hline
\end{tabular}
\end{table}

Итенсивность обслуживания \( \mu \) есть величина обратно пропорциональная периоду следования кадров \( T = 8 * (X + 26) * bt + IFG \).
\[ \mu = \frac{1}{T} = \frac{1}{ 8 * (X + 26) * bt + IFG } \]
\end{block}
\end{frame}

\begin{frame}
\begin{table}[h!]
	\begin{tabular}{|p{0.22\textwidth}|p{0.22\textwidth}|p{0.22\textwidth}|p{0.22\textwidth}|}
	\hline Технология Ethernet & Битовая скорость & Длина кадра (байт) & Интенсивность \( \mu \) (кадр/мс) \\
	\hline \multirow{2}{*}{Fast Ethernet} 	& \multirow{2}{*}{100 Мбит/с} 	& 72 	& 148.800 \\
	\hhline{~~--}				  			& \multirow{2}{*}{}           	& 1526 	& 8.127 \\ 
	
	\hline \multirow{2}{*}{Gigabit Ethernet} 	& \multirow{2}{*}{1 Гбит/с} 		& 72 	& 1488.095 \\
	\hhline{~~--}				  			& \multirow{2}{*}{}           	& 1526 	& 81.274 \\ 
	
	\hline \multirow{2}{*}{10G Ethernet} 		& \multirow{2}{*}{10 Гбит/с} 	& 72 	& 14880.952 \\
	\hhline{~~--}				  			& \multirow{2}{*}{}           	& 1526 	& 812.744 \\ 
	
	\hline \multirow{2}{*}{40G Ethernet} 		& \multirow{2}{*}{40 Гбит/с} 	& 72 	& 59523.800 \\
	\hhline{~~--}				  			& \multirow{2}{*}{}           	& 1526 	& 3250.975 \\ 
	
	\hline \multirow{2}{*}{100G Ethernet} 	& \multirow{2}{*}{100 Гбит/с} 	& 72 	& 148809.524 \\
	\hhline{~~--}				  			& \multirow{2}{*}{}           	& 1526 	& 8127.438 \\
	\hline
	\end{tabular}
\end{table}
\end{frame}

%\begin{frame}
%
%\begin{block}{}
%\end{block}
%\end{frame}

\end{document}