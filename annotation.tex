%TODO: в конечной версии draft заменить final
\documentclass[oneside, final, 14pt, a4paper]{extreport}
\renewcommand{\rmdefault}{ftm} % Times New Roman

\usepackage[utf8]{inputenc}
\usepackage[russianb]{babel}

% красная строка для всех абзацев
\usepackage{indentfirst}

% отступы
\usepackage{vmargin}
\setmarginsrb{3cm}{2cm}{1.5cm}{2cm}{0pt}{0pt}{0pt}{1.3cm}

% полуторный интервал только для текста
\usepackage{setspace}
\onehalfspacing

% --------------------- определение команд рубрикации ---------------------
%\newcommand\Chapter[1]
%{
%	\refstepcounter{chapter}
%	\chapter*
%	{
%		\begin{huge}		
%			\textbf
%			{
%				\raggedright \centering
%				\chaptername\ \arabic{chapter}. #1\\
%			}
%		\end{huge}
%		\bigskip
%	}
%	
%	\addcontentsline{toc}{chapter}{\arabic{chapter}. #1}
%}
%
%\newcommand\Section[1]
%{
%	\refstepcounter{section}
%	\section*
%	{
%		\raggedright \centering
%		\arabic{chapter}. \arabic{section}. #1
%	}
%	
%	\addcontentsline{toc}{section}{\arabic{chapter}. \arabic{section}. #1}
%}
%
%\newcommand\Subsection[1]
%{
%	\refstepcounter{subsection}
%	\subsection*
%	{
%		\raggedright \centering
%		\arabic{chapter}. \arabic{section}. \arabic{subsection}. #1		
%	}
%	
%	\addcontentsline{toc}{subsection}{\arabic{chapter}. \arabic{section}.  \arabic{subsection}. #1}
%}
% -------------------------------------------------------------------------

%% точки в оглавлении
%\usepackage{tocstyle}
%\usetocstyle{allwithdot}
%
%% точка вместо квадратных скобок в списке литературы
%\makeatletter
%\renewcommand*{\@biblabel}[1]{\hfill#1.}
%\makeatother

%TODO: в конечной версии fussy заменить на sloppy
\sloppy

\begin{document}
\pagenumbering{gobble}

\chapter*{\centering Аннотация}
Выпускная квалификацонная работа посвящена разработке программной среды аналитического моделирования информационно-вычислительных сетей.

Стремительное развитие средств связи споровождаются непрерывной сменой сетевых технологий. Для непрерывного количественного и качественного роста компьютерных сетей необходимо развитие теории в этой области и создание методов анализа, направленных на сокращение сроков и повышение качества проектирования компьютерных сетей.

В качестве такой теории выступает теория систем и сетей массового обслуживания.
Математические методы этой теории обеспечивают возможность решения многочисленных задач расчёта
характеристик качества функционирования различных компонентов компьютерных сетей.

В первой главе рассмотрена теория, необходимая для построения математических моделей ИВС и расчёта их вероятностно-временных характеристик и характеристик надёжности.

Во второй главе представлены основные моменты программной реалзиации программной среды моделирования ИВС и некоторые ипользованные алгоритмы.

В третьей главе описаны задачи и методология модельного эксперимента и анализ полученный во время эксперимента результатов.

Работа состоит из введения, трёх глав, девятнадцати параграфов списка литературы.

В выпускной работе содержится 2 таблицы и 21 рисунок. Список литературы содержит 9 источников. Общее количество страниц работы - 49.

\end{document}