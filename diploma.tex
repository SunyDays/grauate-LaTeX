% в конечной версии draft заменить final
\documentclass[oneside, draft, 14pt, a4paper]{extreport}
\renewcommand{\rmdefault}{ftm} % Times New Roman

\usepackage[utf8]{inputenc}
\usepackage[russianb]{babel}
\usepackage{amssymb}
\usepackage{amsmath}

% красная строка для всех абзацев
\usepackage{indentfirst}

% отступы
\usepackage{vmargin}
\setmarginsrb{3cm}{2cm}{1.5cm}{2cm}{0pt}{0pt}{0pt}{1.3cm}

% полуторный интервал только для текста
\usepackage{setspace}
\onehalfspacing

% --------------------- определение команд рубрикации ---------------------
\newcommand\Chapter[1]
{
	\refstepcounter{chapter}
	\chapter*
	{
		\begin{huge}		
			\textbf
			{
				\raggedright \centering
				\chaptername\ \arabic{chapter}. #1\\
			}
		\end{huge}
		\bigskip
	}
	
	\addcontentsline{toc}{chapter}{\arabic{chapter}. #1}
}

\newcommand\Section[1]
{
	\refstepcounter{section}
	\section*
	{
		\raggedright \centering
		\arabic{chapter}. \arabic{section}. #1
	}
	
	\addcontentsline{toc}{section}{\arabic{chapter}. \arabic{section}. #1}
}

\newcommand\Subsection[1]
{
	\refstepcounter{subsection}
	\subsection*
	{
		\raggedright \centering
		\arabic{chapter}. \arabic{section}. \arabic{subsection}. #1		
	}
	
		\addcontentsline{toc}{subsection}{\arabic{chapter}. \arabic{section}.  \arabic{subsection}. #1}
}
% -------------------------------------------------------------------------

% точки в оглавлении
\usepackage{tocstyle}
\usetocstyle{allwithdot}

% точка вместо квадратных скобок в списке литературы
\makeatletter
\renewcommand*{\@biblabel}[1]{\hfill#1.}
\makeatother

% в конечной версии fussy заменить на sloppy
\fussy

\begin{document}

\renewcommand{\contentsname}{\centering Оглавление}
\tableofcontents
\thispagestyle{empty}

\chapter*{\centering Введение}
\addcontentsline{toc}{chapter}{Введение}

\section*{\centering Актуальность}
\addcontentsline{toc}{section}{Актуальность}
Повсеместное внедрение компьютерных сетей, успехи в развитии оптоволоконных и беспроводных средств связи
споровождаются непрерывной сменой сетевых технологий, направленной на повышение быстродействия
и надёжности сетей. Однако создание опытного образца сети для оценки её эффективности не всегда является
оправданным с точки зрения времени и трудоёмкости, поэтому разработка математических моделей является актуальной задачей.

Для непрерывного количественного и качественного роста компьютерных сетей необходимо развитие фундаментальной
теории в этой области и создание инженерных методов анализа, направленных на сокращение сроков и повышение
качества проектирования компьютерных сетей.

В качестве такой теории выступает теория систем и сетей массового обслуживания.
Математические методы этой теории обеспечивают возможность решения многочисленных задач расчёта
характеристик качества функционирования различных компонентов компьютерных сетей.

\section*{\centering Цель}
\addcontentsline{toc}{section}{Цель}
В данной работе рассматривается анализ критериев времени и надёжности доставки информации в информационно-вычислительных
сетях (ИВС) большой размерности различных топологий с множественным методом доступа без коллизий,
построенных на основе технологий семейства Ethernet.

\section*{\centering Задачи}
\addcontentsline{toc}{section}{Задачи}
В задачи исследования входит:
\begin{enumerate}
	\item Изучение методики разработки моделей сетей
	\item Разработка аналитических математических моделей ИВС
	\item Разработка программы для вычисления стационарных и интегральных вероятностных характеристик заданной ИВС
\end{enumerate}


\section*{\centering Методы}
\addcontentsline{toc}{section}{Методы}
Модельный эксперимент и математические модели фрагментов сетей основываются на использованнии математического аппарата
систем и сетей массового обслуживания.

\section*{\centering Значимость}
\addcontentsline{toc}{section}{Значимость}
Разработанная программа автоматизирует рутинную работу по вычислению стационарных и интегральных вероятностных характеристик.
Она будет полезна при:
\begin{itemize}
	\item[-] предварительной оценке характеристик проектируемой сети
	\item[-] оценке характеристик уже существующих сетей
	\item[-] изучении влияния изменений топологии и/или оборудования на характеристики сети
\end{itemize}

\Chapter{<Аналитическая часть>}

\Chapter{Теоретичские основы разработки математической модели ИВС}
\Section{Однородные экспоненциальные сети}
Предметом изучения сетей массового обслуживания (СеМО) являются методы количественного анализа очередей при взаимодействии
множества центров обслуживания и потоков сообщений.

СеМО представляет собой совокупность конечного числа \( M \) обслуживающих центров, в которой циркулируют сообщения,
переходящие в соответствии с маршрутной матрицей (см. \ref{routingMatrix})  из одного центра сети в другой.
Центром обслуживания является система массвого обслуживания, состоящую из \( A \; (1 \leqslant A \leqslant \infty) \) одинаковых приборов
и буфера объёмом \( C \; (0 \leqslant C \leqslant \infty) \). Если в момент поступления сообщения все обслуживающие приборы центра заняты, то сообщение занимает очередь в буфере и ожидает обслуживания \cite[стр. ~90]{vishnevsky}.

В дальнейшем будем пологать, что объём буфера в центре обслуживания \( C = \infty \), время обслуживания заявок распределено по экспоненциальному закону, а распределение входящего потока имеет распределение Пуассона.

СеМО с такими распределениями длительности обслуживания и входящего потока являются однородными экспоненциальными сетями
или сетями Джексона \cite[стр. ~94]{vishnevsky}. Такая модель даёт верхнюю границу оценки (худший вариант)
и стационарные вероятности остояний сети имеют мультипликативную форму.

В данной работе используются открытые сети Джексона, обрабатывающие \( R \) входящих потоков. В открытую сеть сообщения поступают из внешнего источника,
могут покидать сеть после завершения обслуживания и интенсивность входного потока не зависит от состояния сети.

\Subsection{Пуассоновский поток}
Предположение о том, что входящий поток является Пуассоновским, значительно облегчает математические выкладки при достаточной точности.
%\clearpage

\noindent Пуассоновский поток имеет следующие свойства \cite[стр. ~12]{hinchin}:
\begin{enumerate}
	\item Стационарность --- вероятность появления \( k \) событий на любом промежутке времени зависит только от числа \( k \) и от длительности
	\( t \) промежутка.
	
	\item Ординарность --- вероятность наступления за элементарный промежуток времени более одного события мала по сравнению с вероятностью
	наступления за этот промежуток не более одного события и ей можно пренебречь.
	
	\item Независимость --- вероятность появления \( k \) на любом промежутке времени не зависит от того, появлялись или не появлялись
	события в моменты времени, предшествующие началу рассматриваемого промежутка.
\end{enumerate}

\Subsection{Маршрутная матрица}
\label{routingMatrix}
Маршрутна матрица задаёт структуру соединений узлов сети (топологию) и вероятности переходов сообщения из одного центра сети, поле завершения обслуживания
в нём, в другой.
Для открытой сети в качестве внешнего источника вводится новый центр с индексом \( 0 \).
Таким образом маршрутная матрица имеет вид \( P = \: \parallel P_{ij} \parallel \),
где \cite[стр. ~17]{klimanov}:
\\ \( i, j = \overline{0, n}, \: n \) - число узлов в сети,
\\ \( P_{0j} \) - вероятность поступления сообщения в \( M_j \) узел сети из внешнего источника,
\\ \( P_{i0} \) - вероятность покидания сообщением сети после окончания обработки в \( M_i \) узле,
\\ \( P_{ij} \) - вероятность перехода сообщения в узел \( M_j \) после обработки в узле \( M_i \).
\\ \( P_{00} = 0\).

\noindent В маршрутной матрице должно выполняться равенство \( \sum\limits_{j = 0}^{n} P_{ij} = 1, \: i = \overline{1, n} \).
Т.е. событие, состоящие в том, что сообщение после обработки в узле сети перейдёт в другой узел или покинет сеть --- достоверное.

Для сети, обрабатывающей \( R \) входящих потоков, необходимо задать \( R \) маршрутных матриц \( P^{m}, \: m = \overline{1, R} \) для каждого входного потока.

\Section{Методика разработки математической модели ИВС}
Одним из самых распространнёных методов для разработки аналитической математической модели ИВС является прилижённая
декомпозиционная модель сети массового обслуживания, основанная на составлении уравнений баланса средних для класса мультипликативных сетей.
Эта модель допускает простую декомпозицию всей сети на отдельные элементы и обратную операцию - композицию. Такой подход позволяет проводить
анализ каждого фрагмента сети независимо, а затем объединять эти фрагменты, получая обобщённые характеристики.

\noindent Этот метод состоит из следующих этапов:
\begin{enumerate}
	\item  Составление уравнений баланса интенсивностей потоков.
	\item  Вычислеие коэффициентов передачи из уравнений баланса.
	\item  Вычисление стационарных вероятностно-временных характеристик (ВВХ) для каждого отдельного элемента СеМО.
	\item  Вычисиление интегральных ВВХ при взаимодействии двух любых абонентов сети.
\end{enumerate}

Исходными параметрами модели являются интенсивности обслуживающих узлов сети \( \mu_{i}^{m} \), интенсивности поступления требований из внешних иисточников
\( \lambda_{i}^{m} \) и маршрутная матрица \( P^{m} \) для каждого входного потока \( m = \overline{1, R} \).

\Subsection{Уравнения баланса}
Уравнения баланса позволяют найти общие интенсивности потоков \( \lambda_{i}^{'m} \) сообщений в стационарном режиме открытой СеМО (стационарным режимом называется состояние сети,
при \( t \rightarrow \infty \)).

\noindent Эти интенсивности определяются как \( \lambda_{i}^{'m} = e_{i}^{m} \lambda_{0}^{m} \), где:
\\ \( e_{i}^{m} \) - коэффициенты передачи, получаемые при решении уравнений баланса,
\\ \( \lambda_{0}^{m} = \sum\limits_{i = 1}^{n} \lambda_{i}^{m}\) - суммарная интенсивность всех внешних потоков типа \( m \).

Общая интенсивность птоков складывается из интенсивностей поступления требований в \( M_{i} \) узел из внешнего источника \( P_{0i}^{m} \lambda_{0}^{m}, P_{0i}^{m} = \frac{\lambda_{j}^{m}}{\lambda_{0}^{m}} \)
и интенсивностей поступления требований от других узлов \( e_{j}^{m} P_{ji}^{m} \lambda_{0}^{m} \) \cite[стр. ~17]{klimanov}.

\[ e_{i}^{m} \lambda_{0}^{m} = P_{0i}^{m} \lambda_{0}^{m} + \sum\limits_{j = 1}^{n} e_{j}^{m} P_{ji}^{m} \lambda_{0}^{m}, i = \overline{1, n}, m = \overline{1, R} \]

\noindent Видно, что можно сократить обе части уравнения на \( \lambda_{0}^{m} \).

\[ e_{i}^{m} = P_{0i}^{m} + \sum\limits_{j = 1}^{n} e_{j}^{m} P_{ji}^{m}, i = \overline{1, n}, m = \overline{1, R} \]

\[ \Updownarrow \]

\[
 	\left\{
		\begin{aligned}
			& e_{1}^{m} = P_{01}^{m} + e_{1}^{m} P_{11}^{m} + \cdots + e_{n}^{m} P_{n1}^{m} \\
			& \vdots \\
			& e_{n}^{m} = P_{0n}^{m} + e_{1}^{m} P_{1n}^{m} + \cdots + e_{n}^{m} P_{nn}^{m}
		\end{aligned}
	\right.
\]

\noindent После решения этой системы уравнений получаем \( e_{i}^{m} \) для каждого узла, что позволяет рассчитать \( \lambda_{i}^{'m} \).

\Subsection{Коэффициент загрузки}
Коэффициент загрузки для узла \( M_{i} \) вычисляется по формуле
\[ \rho_{i} = \sum\limits_{m = 0}^{R} \rho_{i}^{m}, \: i = \overline{1, n}, \: \rho_{i}^{m} = \frac{\lambda_{i}^{'m}}{\mu_{i}^{m}} \]
%где \( \rho_{i}^{m} = \frac{\lambda_{i}^{'m}}{\mu_{i}^{m}} \)
\( \rho_{i}^{m} \)- коэффициент использования (загрузки) узла , равный отношению общей интенсивности, с которой сообщения поступают в узел, к интенсивности обработки
сообщений этим узлом \cite[стр. ~34]{kleinrock:qs}

И для коэффициента загрузки и для коэффициента использования должно выполняться уловие стационарности для кажддого узла сети
\[ 0 \leqslant \rho_{i}, \rho_{i}^{m} \leqslant 1 \]

\Subsection{Стационарные вероятностно-временные характеристики}


\Subsection{Интегральные вероятностно-временные характеристики}

\Subsection{Вероятность выбора маршрута}

\Chapter{<Проектная часть>}

\chapter*{\centering Заключение}
\addcontentsline{toc}{chapter}{Заключение}

\renewcommand{\bibname}{\centering Список литературы}
\begin{thebibliography}{00}
\addcontentsline{toc}{chapter}{Список литературы}

\bibitem{klimanov} Климанов В.П., Руделёв Р.А.
Моделирование информационных систем. Математические модели для разработки информационных систем: методика и решения: учебное пособие.
--- Москва: ФГБОУ ВПО МГТУ <<СТАНКИН>>, 2014.
--- 45 с.

\bibitem{olifers} Олифер В.Г., Олифер Н.А.
Компьютерные сети. Принципы, технологии, протоколы: учебник для вузов.
--- 4-е изд.
--- Санкт-Петербург: Питер, 2010.
--- 944 с.

\bibitem{tanenbaum} Таненбаум Э., Уезеролл Д.
Компьютерные сети.
--- 5-е изд.
--- Санкт-Петербург: Питер, 2012.
--- 960 с.

\bibitem{vishnevsky} Вишневский В.М.
Теоретические основы проектирования компьютерных сетей: монография.
--- Москва: Техносфера, 2003.
--- 512 с.

\bibitem{pisarev} Писарев В.Н.
Применение теории массового обслуживания в задачах инженерно-авиационного обеспечения.
--- Типография ВВИА имени проф. Н.Е. ~Жукова, 1965.
--- 45 с.

\bibitem{kleinrock:qs} Клейнрок Л.
Теория массового обслуживания.
--- Москва: Машиностроение, 1979.
--- 432 с.

\bibitem{kleinrock:ca} Клейнрок Л.
Вычислительные системы с очередями.
--- Москва: Издательство <<Мир>>, 1979.
--- 600 с.

\bibitem{kleinrock:smfad} Клейнрок Л.
Коммуникационные сети (стохастические потоки и задержки сообщений).
--- Москва: Главная редакция физико-математической литературы изд-ва <<Наука>>, 1970.
--- 256 с.

\bibitem{barashin} Барашин Г.П., Харкевич А.Д., Шнепс М.А.
Массовое обслуживание в телефонии.
--- Москва: Издательство <<Наука>>, 1968.
--- 246 с.

\bibitem{cox} Кокс Д.Р., Смит У.Л.
Теория очередей.
--- Москва: Издательство <<Мир>>, 1966.
--- 218 с.

\bibitem{hinchin} Хинчин А.Я.
Работы по математической теории массового обслуживания
--- Москва: Физматгиз, 1963.
--- 236 с.

\end{thebibliography}

\end{document}