% в конечной версии draft заменить final
\documentclass[oneside, draft, 14pt, a4paper]{extreport}
\renewcommand{\rmdefault}{ftm} % Times New Roman

\usepackage[utf8]{inputenc}
\usepackage[russianb]{babel}

% красная строка для всех абзацев
\usepackage{indentfirst}

% отступы
\usepackage{vmargin}
\setmarginsrb{3cm}{2cm}{1.5cm}{2cm}{0pt}{0pt}{0pt}{1.3cm}

% полуторный интервал только для текста
\usepackage{setspace}
\onehalfspacing

% --------------------- определение команд рубрикации ---------------------
\newcommand\Chapter[1]
{
	\refstepcounter{chapter}
	\chapter*
	{
		\begin{huge}		
			\textbf
			{
				\raggedright \centering
				\chaptername\ \arabic{chapter}. #1\\
			}
		\end{huge}
		\bigskip
	}
	
	\addcontentsline{toc}{chapter}{\arabic{chapter}. #1}
}

\newcommand\Section[1]
{
	\refstepcounter{section}
	\section*
	{
		\raggedright \centering
		\arabic{chapter}. \arabic{section}. #1
	}
	
	\addcontentsline{toc}{section}{\arabic{chapter}. \arabic{section}. #1}
}
% -------------------------------------------------------------------------

% точки в оглавлении
\usepackage{tocstyle}
\usetocstyle{allwithdot}

% точка вместо квадратных скобок в списке литературы
\makeatletter
\renewcommand*{\@biblabel}[1]{\hfill#1.}
\makeatother

% в конечной версии fussy заменить на sloppy
\fussy

\begin{document}

\renewcommand{\contentsname}{\centering Оглавление}
\tableofcontents
\thispagestyle{empty}

\chapter*{\centering Введение}
\addcontentsline{toc}{chapter}{Введение}

\Chapter{<Аналитическая часть>}

\Chapter{<Теоретичская часть>}

\Chapter{<Проектная часть>}

\chapter*{\centering Заключение}
\addcontentsline{toc}{chapter}{Заключение}

\renewcommand{\bibname}{\centering Список литературы}
\begin{thebibliography}{00}
\addcontentsline{toc}{chapter}{Список литературы}

\bibitem{klimanov:manual} Климанов ~В.П., Руделёв ~Р.А.
Моделирование информационных систем. Математические модели для разработки информационных систем: методика и решения: учебное пособие
--- Москва: Издательский центр ФГБОУ ВПО МГТУ Московский госсударственный технологический университет <<СТАНКИН>>, 2014.
--- 46 с.

\bibitem{vishnevksy} Вишневский ~В.М.
Теоретические основы проектирования компьютерных сетей: монография
--- Москва: Техносфера, 2003.
--- 512 с.

\end{thebibliography}

\end{document}
